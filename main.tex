 %%%%%%%%%%%%%%%%%%%% author.tex %%%%%%%%%%%%%%%%%%%%%%%%%%%%%%%%%%%
%
% sample root file for your "contribution" to a contributed volume
%
% Use this file as a template for your own input.
%
%%%%%%%%%%%%%%%% Springer %%%%%%%%%%%%%%%%%%%%%%%%%%%%%%%%%%


% RECOMMENDED %%%%%%%%%%%%%%%%%%%%%%%%%%%%%%%%%%%%%%%%%%%%%%%%%%%
\documentclass[graybox]{svmult}


%\usepackage{mathptmx}       % selects Times Roman as basic font
\usepackage{helvet}         % selects Helvetica as sans-serif font
\usepackage{courier}        % selects Courier as typewriter font
\usepackage{type1cm}        % activate if the above 3 fonts are
                            % not available on your system
%
\usepackage{makeidx}         % allows index generation
\usepackage{graphicx}        % standard LaTeX graphics tool
                             % when including figure files
\usepackage{multicol}        % used for the two-column index
\usepackage[bottom]{footmisc}% places footnotes at page bottom

%%% PUT YOUR DEFINITIONS HERE - BEFORE \begin{document}

\usepackage{amsmath, amssymb}

\usepackage{listings}
\usepackage{lstautogobble}
\usepackage{lstlinebgrd}

\usepackage[scaled]{inconsolata}
\usepackage{xcolor}
\definecolor{keywordcolor}{rgb}{0.7, 0.1, 0.1}   % red
\definecolor{tacticcolor}{rgb}{0.0, 0.1, 0.6}    % blue
\definecolor{commentcolor}{rgb}{0.4, 0.4, 0.4}   % grey
\definecolor{symbolcolor}{rgb}{0.0, 0.1, 0.6}    % blue
\definecolor{sortcolor}{rgb}{0.1, 0.5, 0.1}      % green
\definecolor{attributecolor}{rgb}{0.7, 0.1, 0.1} % red

\def\lstlanguagefiles{lstlean.tex}
% set default language
\lstset{
  language=lean, 
  basicstyle=\ttfamily\footnotesize,
  autogobble=true,
  escapeinside={(*@}{@*)},
  captionpos=t,
  aboveskip=0.5em,
  belowskip=0em}


\begin{document}

\title*{On Proj schemes associated to multigraded rings and their formalization in dependent type theory}
\titlerunning{Proj Construction} 
\author{Arnaud Mayeux}
% Use \authorrunning{Short Title} for an abbreviated version of
% your contribution title if the original one is too long
\institute{Arnaud Mayeux \at Einstein Institute of Mathematics, The Hebrew University of Jerusalem, Givat Ram. Jerusalem,
9190401, Israel
\email{arnaud.mayeux@mail.huji.ac.il}}

%
%
\maketitle


\section*{Introduction}
From the point of view of physics, mathematics provides a rigorous language to describe and think about physical phenomena. 
Until recently, rigour in mathematics relied on human verifications and was not optimal. Recently, efficient tools appeared to ensure rigour in mathematics: formalization theorem prover like Lean \cite{dMal15, dMU21, Av24, Bu24}. 
This is now a recognized field of study that will impact our way of doing mathematics and consequently physics. Combined with generative AI-methods, mechanization of mathematics and basic science is ahead. 
To solve advanced problems, one often need new science and new mathematics unreachable by previous methods. For example current science anticipates that the solar system will not be viable in a finite number of years, and we expect mechanized science will participate in solving the issue of saving humanity before that time (interstellar travels or other solutions). In this note, we report on our recent formalization of the multi-graded proj construction of Brenner-Schroër. This contributes to the emergence of mechanization of mathematics and science. We provide complete data proof of this formalization. The multi-graded Proj construction lies at the crossroads of algebraic geometry and Lie theory, both used fields in theoretical physics.

The Lean4 code of the formalization of the multi-graded Proj construction is around 8000 lines, for 400 lines of \LaTeX  (around 10 PDF pages) \cite{MZ25}.
We use much of the material on the formalisation of scheme theory available in mathlib \cite{Bual22, WZ22}.
Previously, Zhang has formalised the $\mathbf{N}$-graded $\mathrm{Proj}$ construction \cite{Z23} (using a different method).
It would be very nice to see advances in AI-automated formalisation and a more \LaTeX-friendly interface for formalisation.
As of Spring 2025, Copilot (Lean 2025 AI) remained sometimes helpful but insufficiently effective for assisting with the formalisation of algebraic geometry. Copilot was not able to prove that $0 \times 1 = 0$ in a certain non-trivial graded ring. The present work \cite{MZ25}, as new formalised data, will contribute to the training of these already powerful tools.

We now describe the mathematical content of this note. 

Inspired by projective geometry, Grothendieck defined the Proj scheme of an $\mathbf{N}$-graded ring $A$ \cite{Gr61}. This is a well-known construction that every mathematician in scheme theory knows and that is explained in most books about algebraic geometry. This mathematical construction is used and presented in thousands of publications. 
Beyond the definition of Proj, leading treatments also study quasi-coherent sheaves and twisting sheaves on Proj and use Proj to define blowups.  
This construction was formalised by J. Zhang in Lean \cite{Z23}. 
So it appears that this construction is optimal, well-understood and highly verified.  
However, one can ask the following question. Why Grothendieck assume that $A$ is $\mathbf{N}$-graded instead of graded by a general commutative monoid or group? 
Our answer is: there is no reason from the point of view of Grothendieck scheme theory. In fact, Brenner-Schröer defined the Proj of a ring graded by an arbitrary finitely generated abelian group \cite{BS03}.
To our knowledge, as of 2025, textbooks never mentioned the work of Brenner and Schröer.
 Informal versions of the multigraded Proj construction appear implicitly or
explicitly (but without reference to Brenner-Schröer) in various constructions in advanced geometric representation theory (cf. \cite{MR24}).
The first goal of this note is to survey Brenner-Schröer construction and new results \cite{BS03, MR24, MZ25}. The second goal of this note is to explain that most parts of the present material have been formalised in Lean4. Formalization of mathematics gives a revolutionary paradigm. First of all it ensures complete rigor and transparency in mathematical science. Secondly, combined with AI, formalization of mathematics allows mechanized AI-powered rigorous mathematical research. In the second part of this note, we discuss our joint formalization work. The first part was written after the insight offered by formalization. 


\section{Definition of Proj and Behaviors}
This section is based on \cite{BS03, MR24}. The reference \cite{BS03} laid the foundations by introducing the main definition, and \cite{Ma24} provides new results.  We refer to \cite{Ma24} for historical details. We focus on the mathematical content of the Proj construction here.
If a given commutative ring $A$ is $M$-graded for some abelian group $M$, we will say that a submonoid $S$ of $A$ is homogeneous if every element of $S$ is homogeneous. 
In this situation, the localization $A_S$ of $A$ with respect to $S$ is canonically $M$-graded.
Given a graded $A$-module $Q$, one can also consider the localized module $Q_S$, which has a natural structure of a graded $A_S$-module.
Given a homogeneous submonoid $S \subset A$, we will denote by $\underline{S}$ the homogeneous submonoid consisting of homogeneous divisors of elements in $S$. Note that we have a canonical isomorphism of graded rings
 $A_{ \underline{S}} \cong A_{S}.
$
If $Q$ is a graded $A$-module we also have a canonical identification of graded abelian groups 
$
Q_{\underline{S}} \cong Q_S
$
compatible with the actions of $A_{ \underline{S}} = A_{S}$.
Consider an abelian group $M$, and a commutative $M$-graded ring $A$. 

\vspace{0.25\baselineskip}
\noindent \textbf{Definition:}
 Let $S$ be a homogeneous subset of $A$. We denote by $\deg(S)$ the subset of $M$ defined as 
\[\deg(S):= \{ m \in M : \exists s \in S , s  \in A_m \}. \]

 
Note that $\deg(\{0\})=M$.
More generally, if $S$ is a homogeneous submonoid of $A$, then $\deg(S)$ is a submonoid of $M$, also denoted $M[S\rangle$.

\vspace{0.25\baselineskip}
\noindent \textbf{{Definition:}}
Let $S$ be a homogeneous submonoid of $A$. 
 We put $M[S]=M[S \rangle^{\mathrm{gp}}= M[S \rangle- M[S \rangle$, the subgroup of $M$ generated by $\deg(S)$.


\vspace{0.25\baselineskip}
\noindent\textbf{{Definition:}}
 A homogeneous submonoid $S$ of $A$ is called \emph{$M$-relevant} (or just \emph{relevant} if $M$ is clear from the context) if for any $m$ in $M$ there exists $n \in \mathbf{Z}_{> 0}$ such that $nm$ belongs to $M[\underline{S}]$, i.e. if $M/(M[\underline{S}])$ is a torsion abelian group.
 A family $\{a_i : i \in I\}$ of homogeneous elements in $A$ is called $M$-relevant if the submonoid generated by the $a_i$'s is relevant.
 A homogeneous element $a \in A$ is called $M$-relevant if the family $\{a\}$ is $M$-relevant.




Let $S$ be a homogeneous submonoid of $A$. 


\vspace{0.25\baselineskip}
\noindent
\textbf{\textbf{Definition:}}
The degree-$0$ part $(A_S)_0$ of the localization $A_{S}$ is denoted $A_{(S)}$ and is called the \emph{potion} of $A$ with respect to $S$. 
If $Q$ is a graded $A$-module, we will also denote by $Q_{(S)}$ the degree-$0$ part $(Q_S)_0$ of $Q_S$; it admits a canonical structure of an $A_{(S)}$-module.




We have canonical identifications
$A_{(\underline{S})} \cong A_{(S)}$ and $Q_{(\underline{S})} \cong Q_{(S)}$.
If  $\{a_i : i \in I \}$ is a family of homogeneous elements of $A$, we will denote by
$A_{(\{a_i : i \in I \})}$
the potion associated with the submonoid of $A$ generated by $\{a_i : i \in I \}$; in case $\#I=1$ we will write $A_{(a)}$ for $A_{(\{a\})}$.

If $S$ and $T$ are submonoids of $A$, we will denote by $ST$ the submonoid of $A$ generated by $S \cup T$, i.e. $ST = \{st : s \in S, \, t \in T \}$. Of course, $ST$ is homogeneous if $S$ and $T$ are.
The following is the key result that makes the Proj construction work.

\vspace{0.25\baselineskip}
\noindent
\textbf{{Proposition (Magic of potions):}}
Let $S$ and $T$ be homogeneous submonoids of $A$. 
\begin{enumerate}
\item 
\label{it:magic-1}
We have a canonical homomorphism of potion rings 
%\[\chi : 
$A_{(S )} \to  A_{(ST )}$.
%\]
\item 
\label{it:magic-2}
Assume that $S$ is relevant. Fix a subset $T' \subset T$ which generates $T$ as a submonoid of $(A,\times)$ and, for any $t $ in $T'$, fix $n_t \in \mathbf{N}_{>0}$ and $s_t, s_t' \in \underline{S}$ such that $\deg(t^{n_t}) = \deg(s_t)-\deg(s_t')$.
%$t^{n_t} \smile s_t$. 
Then $\frac{t^{n_t} s_t'} { s_t}$ belongs to $A_{(\underline{S} )}=A_{(S)}$. 
 Moreover we have a canonical isomorphism of $A_{(S )}$-algebras between $ A_{(ST )}$ and the localization of $A_{(S )}$ with respect to the submonoid of $A_{(S)}$ generated by $\{\frac{t^{n_t} s_t'} { s_t} : t \in T' \}$.
\item 
\label{it:magic-3}
Assume that $S$ is relevant and that $T$ is finitely generated as submonoid of $(A,\times)$. The morphism of schemes 
$
\mathrm{Spec} (A_{(ST )}) \to \mathrm{Spec} (A_{(S)})
$
induced by the ring homomorphism in (1) is an open immersion of schemes.
\item 
\label{it:magic-4}
Let $f_1, \ldots, f_n \in A$ be nonzero relevant homogeneous elements of the same degree.
Then we have a canonical open immersion
\[
\mathrm{Spec}(A_{ (f_1 + \cdots + f_n) }) \to \mathrm{Spec} (A_{(f_1)}) \cup \cdots \cup \mathrm{Spec} (A_{(f_n)})
\]
where the right-hand side is defined as the glueing of the affine schemes $\mathrm{Spec} (A_{(f_i)})$ along the open subschemes $\mathrm{Spec} (A_{(f_i \cdot f_j)}) \subset \mathrm{Spec} (A_{(f_i)})$.
\end{enumerate}




From now on we assume that $M$ is a \emph{finitely generated} abelian group, and fix a commutative $M$-graded ring $A$.

We will denote by $\mathcal{F}_A$ the set of all relevant homogeneous submonoids of $A$ which are finitely generated as submonoids of $(A,\times)$.

\vspace{0.25\baselineskip}
\noindent
\textbf{{Construction (Proj as glueing potions)}}
Let
$\mathcal{F} \subset \mathcal{F}_A$ be a subset.
% of relevant homogeneous submonoids of $A$ such that any element in $\mathcal{F}$ is finitely generated as a submonoid of $(A, \times)$. 
 For each $S \in \mathcal{F}$, let $D_{\dagger}(S)$ be the spectrum of the potion $A_{(S)}$. 
 If $S,T \in \mathcal{F}$, the affine scheme $D_{\dagger}(ST)$ identifies canonically with an open subscheme of $D_{\dagger}(S)$. For each $S,T \in \mathcal{F}$, we have equalities
$
 D_{\dagger}({S S}) = D_{\dagger}(S) $ and $ D_{\dagger}({ST})=D_{\dagger}({TS}).
$
 Moreover, for each triple $S,T,U \in \mathcal{F}$, we have
$
 D_{\dagger}({ST} )\cap D_{\dagger}({SU}) = D_{\dagger}({TS}) \cap D_{\dagger}({TU}).
$
Now, by glueing, from these data we obtain a scheme $\mathrm{Proj}^M_{\mathcal{F} } (A)$ and, for each $S \in \mathcal{F}$, an open immersion $\varphi_S : D_{\dagger}(S) \to \mathrm{Proj}^M_{\mathcal{F}} (A)$, such that
$
\mathrm{Proj}^M_{\mathcal{F} } (A) = \bigcup_{S \in \mathcal{F} } \varphi_S( D_{\dagger}(S)).
$
In practice, we will often identify $D_{\dagger}(S) $ and $\varphi_S (D_{\dagger}(S))$. 
In the case when $\mathcal{F} = \mathcal{F}_A$,
 the scheme $\mathrm{Proj}^M_{\mathcal{F}_A} (A)$ will be denoted $\mathrm{Proj}^M(A)$, or just $\mathrm{Proj}( A)$ when $M$ is clear from the context.






\vspace{0.25\baselineskip}
\noindent
\textbf{{Proposition:}}
The scheme $\mathrm{Proj}^M(A)$ is quasi-separated.



\vspace{0.25\baselineskip}
\noindent
{\textbf{Proposition (Functoriality of Proj):}} 
Let $\Psi:A \to B $ be a homomorphism of $M$-graded rings. 
For any $\mathcal{F} \subset \mathcal{F}_A$ 
%be a set of finite relevant families of $A$. Then $\Psi (\mathcal{F})$ is a set of relevant families of $B$ and 
we have a canonical morphism of schemes $\mathrm{Proj}^M_{\Psi (\mathcal{F})} (B) \to \mathrm{Proj}^M_{\mathcal{F} } (A)$. Moreover, for any $S \in \mathcal{F}$ we have
$\mathrm{Proj}^M_{\Psi(\mathcal{F})}(B) \times_{\mathrm{Proj}_{\mathcal{F}}^M(A)} D_\dag(S) = D_\dag(\Psi(S));$
in particular, the morphism is affine.



\vspace{0.25\baselineskip}
\noindent
\textbf{{Proposition:}}
Assume that $\Psi : A \to B$ is surjective. Then we have $\mathrm{Proj}^M_{\Psi (\mathcal{F}_A)} (B) = \mathrm{Proj}^M (B)$, and the canonical morphism
 $\mathrm{Proj}^M(B) \to \mathrm{Proj}^M(A)$
is a closed immersion.



\vspace{0.25\baselineskip}
\noindent
\textbf{{Proposition:}}
 Let $M$ and $M'$ be two finitely generated abelian groups. Let $R$ be a commutative ring and let $A$ (resp.~$A'$) be a commutative $M$-graded (resp.~$M'$-graded) $R$-algebra. 
 Then for the natural $(M \times M')$-grading on $A \otimes_R A'$, we have a canonical isomorphism
 \[
 \mathrm{Proj} ^{M \times M'} (A \otimes_R A') \cong \mathrm{Proj} ^M (A) \times_{\mathrm{Spec}(R)} \mathrm{Proj} ^{M'} (A'). 
 \]


We now study quasi-coherent sheaves on Proj.
Let $Q$ be an $M$-graded $A$-module. For any homogeneous submonoid $S \subset A$, we have considered the $A_{(S)}$-module $Q_{(S)}$. The following fact is immediate by glueing of quasi-coherent sheaves.

\vspace{0.25\baselineskip}
\noindent
\textbf{{Definition:}}
There exists 
a unique quasi-coherent $\mathcal{O}_{\mathrm{Proj}^M(A)}$-module $\widetilde{Q}$ such that
$\Gamma \bigl( D_\dag(S)  , \widetilde{Q} \bigr) = Q_{(S)}$
for every $S \in \mathcal{F}_A$.



An $M$-graded $A$-module $Q$ will be called \emph{negligible} if $\widetilde{Q}=0$.



We now define the versions in our setting of the twisting sheaves. 
If $Q$ is a graded $A$-module and $\alpha \in M$, we will denote by $Q(\alpha)$ the $M$-graded $A$-module which coincides with $Q$ as an $A$-module, but with the $M$-grading defined by $(Q(\alpha))_\beta = Q_{\alpha+\beta}$ for $\beta \in M$.


\vspace{0.25\baselineskip}
\noindent
{\textbf{Definition (Twisting sheaves)}}
Let $\alpha \in M$.
\begin{enumerate}
\item
The quasi-coherent sheaf $\widetilde{A(\alpha)}$ on $\mathrm{Proj}^M(A)$ is 
%called the \emph{$\alpha$-th-twist of the structure sheaf} of $\mathrm{Proj}^M(A)$, and is 
denoted $\mathcal{O}_{\mathrm{Proj}^M(A)} (\alpha)$.
\item
If $\mathcal{Q}$ is a sheaf of $\mathcal{O}_{\mathrm{Proj}^M(A)}$-modules, we set $\mathcal{Q} (\alpha) = \mathcal{O}_{\mathrm{Proj}^M(A)} (\alpha ) \otimes_{\mathcal{O}_{\mathrm{Proj}^M(A)}} \mathcal{Q}$.
\end{enumerate}




\vspace{0.25\baselineskip}
\noindent
\textbf{{Proposition:}}
Assume that $A$ is a noetherian ring.
\begin{enumerate}
\item
For any $\alpha \in M$, the quasi-coherent sheaf $\mathcal{O}_{\mathrm{Proj}^M(A)} (\alpha)$ is coherent.
\item
If $Q$ is a finitely generated $M$-graded $A$-module, then $\widetilde{Q}$ is coherent.
\end{enumerate}




A family $S \in \mathcal{F}_A$ will be called \emph{maximally relevant} if $M[\underline{S}]=M$.   We will denote by $\mathcal{F}_A^{\mathrm{m}} \subset \mathcal{F}_A$ the subset consisting of maximally relevant families. In this subsection we will explore various consequences of the following condition:
$
\mathrm{Proj}^M(A) = \bigcup_{S \in \mathcal{F}_A^{\mathrm{m}}} D_\dag(S).$



\vspace{0.25\baselineskip}
\noindent
\textbf{{Proposition:}}
Assume that the above condition is satisfied. Then for any $ \alpha \in M$, the quasi-coherent $\mathcal{O}_{\mathrm{Proj}^M(A)}$-module $\mathcal{O}_{\mathrm{Proj}^M(A)} (\alpha)$ is an invertible sheaf.



\vspace{0.25\baselineskip}
\noindent
\textbf{{Proposition:}}
Assume that the above condition is satisfied, and moreover that $\mathrm{Proj}^M(A)$ is quasi-compact. Then the functor
\[
\mathsf{L} : \mathrm{Mod}^M(A) / \mathrm{Mod}^M(A)_{\mathrm{neg}} \to \mathrm{QCoh}(\mathrm{Proj}^M(A))
\]
is an equivalence of categories.





\vspace{0.25\baselineskip}
\noindent
\textbf{Example (Flag variety as Proj)}
 Let $k$ be an algebraically closed field and let $G$ be
a connected reductive group scheme over $k.$ Let $T$ be a maximal split torus
and $B$ be a Borel subgroup such that $B = T N$ where $N$ is the unipotent
radical. Then $G/N$ is quasi-affine and the ring $A := \Gamma (G/N, \mathcal{O}_{G/N} )$ is canonically $X^*(T)$-graded. We have a schematically dominant open immersion
$G/N \to \mathrm{Spec}(A)$. 
The flag variety $G/B$ identifies with $\mathrm{Proj}^{X^*(T)} (A)$.
Consider a finite-dimensional $G$-module $\widetilde{V}$ and a $B$-stable subspace $V \subset \widetilde{V}$. 
We can then consider the induced scheme
$
 G \times^{B} V,
$
i.e.~the quotient of the product $G \times V$ by the (free) action of $B$ defined by $b \cdot (g,x) = (gb^{-1}, b \cdot x)$.  It is a vector bundle over $G/B$.
There is a canonical way to see such schemes as Proj schemes.
An interesting case is when $\widetilde{V}$ is the Lie algebra of $G$ and $V$ is the Lie algebra of the unipotent radical of $B$. In this case, $G \times^{B} V$ is the so-called \emph{Springer resolution}.


\section{Formalisation}



\subsection{Examples of Formalisation} 

\paragraph{Potions:}%
Compared to sets and subsets, quotient type is a fundamental concept in the type system of Lean4. 
Therefore, it is more natural and ergonomic to define the degree-$0$ part of a graded ring as a quotient type 
instead of a subset of the graded ring. For an $\iota$-graded ring $A \cong \bigoplus_{i : \iota} \mathcal{A}_i$ and a submonoid $x \le A$, 
we define the degree-$0$ part of the localization $A_x$ to be the equivalence classes of triples $(i, n, d)$ where 
$i : \tau$ and both $n$ and $d$ are elements of $\mathcal{A}_i$ under the equivalence relation 
$(i, n, d) \sim (i', n', d')$ if and only if the equality $\frac{n}{d} = \frac{n'}{d'}$ in the localized ring $A_x$:
\begin{lstlisting}
structure NumDenSameDeg where
  deg : ι
  (num den : 𝒜 deg)
  den_mem : (den : A) ∈ x

def NumDenSameDeg.embedding (p : NumDenSameDeg 𝒜 x) :=
  Localization.mk p.num ⟨p.den, p.den_mem⟩

def HomogeneousLocalization : Type _ :=
  Quotient (Setoid.ker <| NumDenSameDeg.embedding 𝒜 x)
\end{lstlisting}
The advantage of the quotient approach is two-fold:
\begin{itemize}
  \item The numerator $n$, the denominator $d$, the degree $i$ and the fact that $n, d \in \mathcal{A}_i$ and $d \in x$ and  are one application of axiom of choice away.
  Compared to the set-theoretic approach $A_{(x)} = \left\{f\middle|\exists i \in \iota, n \in \mathcal{A}_i \cap x, d \in \mathcal{A}_i, f = \frac{n}{d}\right\}$, the extraction of the data needs repeated application of AC.
  \item The quotient type comes with a universal property already making defining functions out of $A_{(x)}$ convenient.
\end{itemize}

\paragraph{Magic of Potions:}%
The practice of formalisation often forces a clearer statement.
In the second part of the magic of potions, we can define the following data type:
\begin{lstlisting}
structure PotionGen where
  (index : Type*)
  (elem : index → A)
  (elem_mem : ∀ t, elem t ∈ T)
  (gen : Submonoid.closure (Set.range elem) = T.toSubmonoid)
  (n : index → ℕ+)
  (s s' : index → A)
  (s_mem_bar : ∀ t, s t ∈ S.bar)
  (s'_mem_bar : ∀ t, s' t ∈ S.bar)
  (i i' : index → ι)
  (t_deg : ∀ t : index, (elem t : A)^(n t : ℕ) ∈ 𝒜 (i t - i' t))
  (s_deg : ∀ t, s t ∈ 𝒜 (i t))
  (s'_deg : ∀ t, s' t ∈ 𝒜 (i' t))

def PotionGen.genSubmonoid (T' : PotionGen S T) : Submonoid S.Potion :=
  Submonoid.closure
    {x | ∃ (t : T'.index), x =
      S.equivBarPotion.symm (.mk
        { deg := T'.i t,
          num := ⟨(T'.elem t) ^ (T'.n t : ℕ) * T'.s' t,
            by simpa using SetLike.mul_mem_graded (T'.t_deg t) (T'.s'_deg t)⟩,
          den := ⟨T'.s t, T'.s_deg t⟩,
          den_mem := T'.s_mem_bar t }) }
\end{lstlisting}
Therefore, the statement of the second part of the magic of potions can be stated as:
for every finitely generated, relevant homogeneous submonoid $S, T$ of $A$ and any \lstinline|PotionGen S T| $T'$, we have an isomorphism between
the potion $A_{(ST)}$ and the localized ring of $A_S$ at \lstinline|genSubmonoid| $T'$:
\begin{lstlisting}
def localizationRingEquivPotion (T' : PotionGen S T) :
    Localization T'.genSubmonoid ≃+* (S * T).Potion := ...
\end{lstlisting}
Of course, in a pen-and-paper proof, one could define auxiliary structures to make the same effect, 
but the absence of such structures does not put the burden to the author, therefore, the author is 
less motivated to define such structures. However, in formalisation, 
the burden is put on the authors --- without such structures, the proof is not only less readable to the reader, but much harder to write for the author as well.

% \paragraph{Commutative diagram:}%
% {\bf to be written}

\paragraph{In pursuit of absolute rigour}%
On the other hand, the clarity and rigour brought by formalisation \textit{a priori} requires extra 
care from the formaliser --- steps seen as unnecessary on pen-and-paper suddenly need to 
be explained to computers. One can admit that this is because type theory is not the same as set theory and 
one should not expect verbatim practices and there are many places where intermediate steps are indeed missing.
A noticeable difference is the notion of equality --- consider the following example, if $S$ and $T$ are two equal homogeneous submonoids $A$, 
we should be able to treat $A_{(S)}$ and $A_{(T)}$ as the same object in some sense. 
Prior to formalisation, we thought $A_{(S)}$ and $A_{(T)}$ are equal, but in set theory, $A_{(S)}$ and $A_{(T)}$ are subrings of $A_S$ and $A_T$ respectively,  
therefore, the literal equality $A_{(S)} = A_{(T)}$ is simply not true. 
In type theory, though we can prove the equality $A_{(S)} = A_{(T)}$ holds, the equality provide less useful functions because of the excessive need of \lstinline|rewrite| of the equality $S = T$.
Therefore, $A_{(S)}$ and $A_{(T)}$ are the same because the following isomorphism, note that both directions of the isomorphism are constructed
using the universal property of quoteint types and not directly from the equality $S = T$:
\begin{lstlisting}
def potionEquiv {S T : HomogeneousSubmonoid 𝒜} (eq : S = T) : 
    S.Potion ≃+* T.Potion := 
  RingEquiv.ofHomInv
    (HomogeneousLocalization.map _ _ (RingHom.id _) ... : 
      S.Potion →+* T.Potion)
    (HomogeneousLocalization.map _ _ (RingHom.id _) ... : 
      T.Potion →+* S.Potion)
    ...
\end{lstlisting}
The requirement of spelling out ``trivial'' isomorphisms is not only an artifact of type theory, 
it is needed for rigour: consider the isomorphism between Cartisean product $A \times B \cong B \times A$, in case of $A = B$,
without more details $A \times A \cong A \times A$ is ambiguous. 


\section{Conclusion}
\begin{tabular}{|l|c|r|}
  \hline
  \textbf{Property of the Proj construction} & \cite{Gr61}  & \cite{BS03,MR24} \\
  \hline
  Grading & $\mathbf{N} $ & finitely genenerated ab. group \\
  \hline
  Definition via prime ideals works & Yes & No\\
   \hline
  Definition via prime ideals required & No & No\\
  \hline
  Definition via glueing works & Yes & Yes\\
  \hline 
  Glueing steps required & Yes & Yes \\ 
  \hline
  Twisting sheaves available & Yes & Yes \\ 
  \hline
 Compatible with tensor product & No & Yes \\ 
  \hline
  Separated & Yes & No, but quasi-separated \\
  \hline 
  Compatible with blowups & Yes & Yes \\
  \hline 
  Allows to describe G/B canonically  & No & Yes \\
  \hline 
  Qcoh on Proj via graded modules   & Yes & Yes \\
  \hline 
  Verified in Lean    & Yes & Yes \\
  \hline 
\end{tabular}
Finally, the level of technicality in the Grothendieck setting is almost the same as that in the Brenner–Schröer setting.
 \begin{thebibliography}{99.}


\bibitem[Av24]{Av24} J.\,Avigad: {\it Automated reasoning for mathematics,} in {\it Automated reasoning. Part I}, 3--20, Lecture Notes in Comput. Sci. Lecture Notes in Artificial Intelligence, 14739 , Springer, Cham, 2024.

\bibitem[BS03]{BS03} H.\,Brenner, S.\,Schröer: {Ample families, multihomogeneous spectra, and algebraization of formal schemes} Pacific J. Math. 208 (2003), no. 2, 209–230.

\bibitem[Bual22]{Bual22} K.\,Buzzard, C.\,Hughes, K. Lau, A\,
              Livingston, R. F. Mir, and S.\,Morrison: {\it Schemes in Lean,} Exp. Math. {\bf 31} (2022), no.~2, 355--363.
                 
              
              
\bibitem[Bu24]{Bu24} K.\,Buzzard: {\it Mathematical reasoning and the computer,} Bull. Amer. Math. Soc. (N.S.) {\bf 61} (2024), no.~2, 211--224.  



\bibitem[Gr61]{Gr61} A.\,Grothendieck: {Éléments de géométrie algébrique : II. Étude globale élémentaire de quelques classes de morphismes} Publications mathématiques de l’I.H.É.S., tome 8 (1961), p. 5-222

\bibitem[Mathlib]{Mathlib} Lean Community: {Mathlib}, https://leanprover-community.github.io/mathlib-overview.html



\bibitem[Ma24]{Ma24} A.\,Mayeux: {\it Multi-centered dilatations, congruent isomorphisms and Rost double deformation space,}
Transformation Groups, 2024 https://doi.org/10.1007/s00031-024-09894-9



\bibitem[MR24]{MR24}A. Mayeux and S. Riche, On multi-graded Proj schemes, to appear in Publ. Res. Inst. Math. Sci. (Kyoto Univ., EMS Press)



\bibitem[MZ25]{MZ25}A. Mayeux and J. Zhang, Multi-graded Proj construction in Lean4, https://github.com/ProjConstruction/Proj (2025)

\bibitem[dMU21]{dMU21}
L.\,de\,Moura and S.\,Ullrich: {\it The Lean 4 theorem prover and programming language,} in {\it Automated deduction---CADE 28}, 625--635, Lecture Notes in Comput. Sci., 12699, Springer, Cham, 2021.

\bibitem[dMal15]{dMal15}
L.\,de\,Moura, S.\,Kong, J.\,Avigad, F.\,van\,Doorn and J.\,von\,Raumer: {\it The lean theorem prover (system description)}, in {\it Automated deduction---CADE 25}, 378--388, Lecture Notes in Comput. Sci. Lecture Notes in Artificial Intelligence, 9195 , 2015 Springer, Cham.

\bibitem[Z23]{Z23}J. Zhang, Formalising the Proj construction in Lean, in {\it 14th International Conference on Interactive Theorem Proving}, Art. No. 35, LIPIcs. Leibniz Int. Proc. Inform., 268 (2023).

\bibitem[WZ22]{WZ22}
E.\,Wieser and J.\,Zhang: {\it Graded rings in Lean's dependent type theory,} in {\it Intelligent computer mathematics}, 122--137, Lecture Notes in Comput. Sci. Lecture Notes in Artificial Intelligence, 13467 , Springer, Cham, 2022.

\end{thebibliography}

%\input{referenc}
\end{document}


